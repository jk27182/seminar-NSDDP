\section{Conclusion}
This seminar paper presented an overview of the field stochastic optimization, what frameworks exist and what advantages and disadvantages each framework provide.
The current state-of-the-art algorithms were presented and the advantages and disadvanteges were discussed.
The main disadvantages that was common for all but approach, was the bad scalability for larger problems with more stages.
The stochastic dual dynamic programming algorithm was able to solve this shortcoming by sampling the scenario tree.
However, this approach also suffered from bad scalability as the number of necessary iteration for SDDP increased exponentially with the dimension of the decision variables.
The neural stochastic dual dynamic programming approach tries to solve these shortcomings by predicting optimality cuts based on a representation of the problem instance with a meta-machine learning mode, which can be used as a warm start for the SDDP algorithm and also by providing a projection to a low-dimensional representation that can be solved faster.
Own Experiments were provided that showcased the advantages and the potential of NSDDP but also the drawbacks of that approach.
These drawbacks where discussed and and outlook for possible future research topic to solve these drawbacks was given.
