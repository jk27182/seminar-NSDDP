\section{Introduction} \label{Introdcution}
Optimization problems can be found in a variety of fields and have multiple applications, for instance in the energy sector \cite{PereiraPinto1991}, logistics, machine learning and many more.
In several industrial settings, the problem is decomposed into multiple stages.
% These stages are usually determined by the  like multiple time periods
These stages are usually defined in terms of time, but are not bound to a specific number of days or hours. For example, a problem that lasts for more than a year, can still be decomposed into two stages if the problem structure only changes once.
These problems arise in a multitude of settings like fleet management \cite{example_powell_fleet_management}, supply-chain optimization \cite{example_supply_chain_optimization}, portfolio optimization \cite{example_portfolio_opt} and power scheduling in a Hydro-Thermal System \cite{example_power_scheduling_Nowak2000}.
These problems involve making decisions based on current knowledge.
After the execution of the first-stage decision, new information from the environment is observed and is used for a successive decision.
Optimization problems from this kind of problem class are called sequential decision problems \cite{Powell_solving_Curses_of_Dimensionality}.\\
% The problem class of these kind of optimization problems are called sequential decision problems \cite{Powell_solving_Curses_of_Dimensionality}.\\
Since these sequential decision problems are calculating decisions over a time frame,
it would be a strong assumption to consider them deterministic, hence they are assumed as stochastic.
One approach to solving optimization problems under uncertainty is stochastic optimization.
According to Powell \cite{POWELL2019795,Powell_Clearing_the_Jungle_of_stochastic_Optimization}, stochastic optimization has many research streams, the major ones being stochastic dynamic optimization and stochastic programming.
Stochastic programming is more common in operations research fields, while stochastic dynamic programming is used in optimal control, like in electrical engineering.
Both frameworks will be briefly discussed in this seminar paper.

This paper discusses the new advances made in the paper 'Neural Stochastic Dual Dynamic Programming' by Dai and Xue et al. \cite{NSDDP}.\\
First, a quick introduction is given to describe different frameworks in order to solve sequential optimization problems, namely stochastic programming and stochastic dynamic programming.
After that, a short overview of the major challenges of these methods is presented, namely the curse of dimensionality.
In the next section, the current state-of-the-art approaches out of the framework of stochastic programming for solving these problems are
presented, namely the L-shaped method, an extension of Benders decomposition, and stochastic dual dynamic programming.
For each approach, the pros and cons are discussed.
Moreover, the neural stochastic dual programming approach from Dai and Xue et al. \cite{NSDDP} is presented in detail with an emphasis on the advantages and disadvantages over current solving methods, and also what the limitations and constraints of this approach are.
Following that section, own computations based on my implementation of the neural stochastic dual dynamic programming approach are presented.
Finally, a conclusion and an outlook on possible future research is given.

\subsection{Stochastic Optimization}
In the following section, the frameworks of stochastic programming and stochastic dynamic optimization are discussed.
They are similar in that they both try to find an optimal policy instead of finding an optimal solution.
An optimal policy is a function that provides an optimal solution given the current known information about the environment \cite{POWELL2019795}. 
The disparities and similarities of these two approaches will be discussed in the following sections.
\subsubsection{Stochastic Programming}\label{Stochastic Programming}
Stochastic programming is a framework from the field of operations research, which deals specifically with the solution of optimization problems that span over multiple periods and involve uncertainty. \\
Stochastic optimization problems, also often referred to in the literature as \glqq Stochastic Program\grqq{} or simply SP, are according to \cite{BirgeLouveaux} optimization problems where part of the information about the problem is uncertain. \\
A common approach to solve this class of problems is to split these problems into two time-discrete stages.
In the first stage, the problem is considered without the knowledge of the uncertain data in the future and is hence deterministic. However, the decision in the first stage, usually denoted by $x$, tries to hedge against future outcomes or possibly occurring future costs, by accounting for these in the objective function. \\
One method to do that is to measure the uncertainty of these stochastic future outcomes with an uncertainty measure \cite{Fuellner_SDDP_TUT}. 
The uncertainty measure of choice is usually the expectation, however other measures are also possible. For a detailed discussion of these measures, see \cite{BirgeLouveaux}.

After the first-stage decision is taken, new information $\omega$ from the environment is observed as a realization of a random variable $\xi$ that represents the stochastic process, where $\omega \in \Omega$ and $\Omega$ is the set of all possible outcomes. \\
That additional information can be used to derive a subsequent second-stage decision as a recourse, usually denoted by $y$ to account for the changes in the environment by which the first stage decision $x$ is no longer optimal. \cite{Fuellner_SDDP_TUT, BirgeLouveaux}.
Since $y$ is dependent on the realization $\omega$, the second stage decision is sometimes denoted as $y(\omega)$.

According to \cite{Lectures_on_stochastic_Programming_Shapiro_Ruszczynski}, the linear case of the described two-stage stochastic optimization problem can be formulated as follows
\begin{subequations}\label{stochastischeFormulierung}
    \begin{alignat}{3}
          &\min_x        &\quad& c^T x + \mathbb{E}_\xi \left[ Q(x,\xi(\omega))  \right] \label{SP_Zielfunktion mit EW}\\
          &\textrm{ s.t.}  &\quad& A x = b \\
          &              &\quad&x\geq0,
    \end{alignat}
\end{subequations}
where the function $Q$ is called the value function of the second stage and according to \cite{Lectures_on_stochastic_Programming_Shapiro_Ruszczynski} is given as
\begin{subequations}\label{Zweite_Stufe_Problem}
    \begin{alignat}{4}
         Q(x,\xi(\omega)) := & \min_y        &\quad& q^T(\omega)y  \\
                             & \textrm{ s.t.}  &\quad&W(\omega)y = h(\omega) - T(\omega)x \label{Recourse Gleichung}\\
                             &                &\quad&y\geq0.
    \end{alignat}
\end{subequations}
In most cases, the underlying stochastic process is represented as a scenario tree, where a sequence of realizations $\{\xi_t\}_{t=1}^{T}$ is called a scenario. \\
An example of a scenario tree for a problem with three stages and three possible stochastic outcomes at the first stage and two possible scenarios at the second stage is shown in figure \ref{fig:scenario_tree}.
This scenario tree has six different scenarios, one possible scenario, marked as \textit{scenario 6}, is highlighted in black.  \\
% Set the overall layout of the tree
\tikzstyle{level 1}=[level distance=3.5cm, sibling distance=3.5cm]
\tikzstyle{level 2}=[level distance=3.5cm, sibling distance=2cm]

% Define styles for bags and leafs
\tikzstyle{end} = [minimum width=3pt]
\tikzstyle{standard} = [circle,draw, minimum width=8pt,fill=gray]
\tikzstyle{scenario} = [circle,draw, minimum width=8pt,fill=black]
\begin{figure}[h]
    \centering
    \begin{tikzpicture}[grow=right, sloped, scale=0.7]
    \node[scenario] {}
        child [black] {
            node[scenario] {} % This is the first of three "Bag 2"
            child {
                node[scenario] {}
                child {
                    node[end] {Scenario 6}
                }
            }
            child [black] {node[standard] {}}
        }
        child [black] {
            node[scenario] {}
            child {
                node [scenario] {}
                    child {
                        node [end] {Scenario 3}
                    }
            }
            child {
                    node[standard] {}
            }
        }
        child [black] {
            node [scenario] {}
                child {% Here are three children, hence three end branches
                    node [standard] {}
                }
                child {
                    node [scenario] {}
                    child {
                        node [end] {Scenario 1}
                    }
                }
        }
    ;
    \end{tikzpicture}
    \caption{Scenario tree with two stages and six scenarios, inspired by \cite{Fuellner_SDDP_TUT, Powell_Clearing_the_Jungle_of_stochastic_Optimization}}
    \label{fig:scenario_tree}
\end{figure}
According to Powell \cite{Powell_solving_Curses_of_Dimensionality}, the sequence of decisions and events can then be stated as follows:
\begin{align*}
    x \rightarrow \xi(\omega) \rightarrow y(\omega,x) .
\end{align*}
The classical formulation of a stochastic optimization problem with recourse assumes a linear objective function, although extensions to the nonlinear case are also possible \cite{BirgeLouveaux}.
The principle of two-stage optimization can be easily transferred to the multistage case, but it introduces additional complexity. \\
According to \cite{Lectures_on_stochastic_Programming_Shapiro_Ruszczynski}, the multistage problem can be formulated as follows.

Let $\xi_t$ be a known random vector with a corresponding population $\Omega_t$ for the stage $t = 2, \dots,T$. $\Omega_t|\omega_{t-1}$ is the conditional population for the stage $t$, given that in stage $t-1$ the event $\omega_{t-1} \in \Omega_{t-1}$ has occurred. \\
A multistage optimization problem with time horizon $T > 2$ has the following structure, again restricting ourselves to the linear case:
\begin{subequations}\label{stochastische Multistage Formulierung}
    \begin{alignat}{2}
         \underset{x}{\min}  & \quad  c_1^T x_1  + \mathbb{E}_{\xi_2} [  \min c_2(\omega_2)x_2(\omega_2) + \mathbb{E}_{\xi_3}[\dots +                                                                                                 \mathbb{E}_{\xi_T}[ \min c_T(\omega_T)x_T(\omega_T)] ]  ]\\
      \textrm{s.t.}  & \quad W_1 x_1 = h_1                                                             \\
                     & \quad T_1 (\omega_2) x_1 + W_2 (\omega_2) x_2(\omega) =h_2(\omega_2),  \quad  \forall \omega_2 \in \Omega_2\\
                     & \quad \quad \vdots                                                                          \\
                      \begin{split}
                        \quad T_{T-1} (\omega_T) x_{T-1} + W_T (\omega_T) x_T(\omega)=h_T (\omega_T),   \\ 
                        \forall \omega_{T-1} \in      \Omega_{T-1},\, \omega_T \in \Omega_T \mid \omega_{T-1} 
                       \end{split} \\
                     &\quad x_1 , x_t(\omega_t) \geq 0                                                \quad   t= 2,\dots,T
    \end{alignat}
\end{subequations}
Problem \ref{stochastische Multistage Formulierung} is also called the extensive form of a stochastic program.
This is in fact the deterministic equivalent of the stochastic program, since for every possible scenario, an extra constraint is added to the problem \cite{BirgeLouveaux}.\\
This is a linear problem that can be solved by conventional solvers for LPs.
However, since the problem size can drastically increase if more complexities are added, large-scale optimization methods should be considered to solve this problem in reasonable computing time. \\
A further constraint is that the number of variables and constraints grow exponentially with an increasing number of stages.
Furthermore, this problem can only be solved for a discrete probability distribution, otherwise the problem has to be discretized, since problem \ref{stochastische Multistage Formulierung} would have an infinite number of constraints \cite{SDDP_Solver_Paper}.

 \subsubsection{Stochastic dynamic optimization} \label{stochastic_dynamic_programming}
Another research stream that focuses on solving multistage stochastic optimization problem is stochastic dynamic optimization.
This research stream is more focused on representing dynamic behavior in an optimization problem and on solving sequential decision problems by solving the well-known Bellman-Equation, which was first introduced by Bellman in 1957 \cite{Bellman1957}. \\
As before, the goal in this framework is not find a single optimal solution vector, but to find an optimal function or policy, that provides an optimal solution given an environmental state $S_t$ at time $t$.
According to Powell \cite{Powell_solving_Curses_of_Dimensionality}, this state describes all information that is necessary to make an optimal decision and is known at stage $t$.
The problem of finding this optimal policy can be traced back to solving the Bellman equation, according to \cite{Einfuehrung_in_das_OR}.
In e.g. \cite{Einfuehrung_in_das_OR} it has already been shown that the determination of an optimal policy or strategy and the optimal value of the problem can be traced back to the solution of the Bellman equation.
This equation has, according to \cite{Powell_solving_Curses_of_Dimensionality}, the form shown in equation \ref{Bellman-Gleichung} and provides an optimality condition for the value functions,
\begin{align}\label{Bellman-Gleichung}
       V_{t}(S_{t})& = \underset{x_{t}\in{\mathcal{X}_{t}}}{\min}\, C_{t}(S_{t},x_{t}) + \gamma \mathbb{E}\left[ V_t(S_{t+1}) \mid S_t \right]
\end{align}
with $V_t(S_t)$ being the value functions, describing the value of a state $S_t$ at a time $t$, provided an optimal strategy is followed from time $t$ \cite{Einfuehrung_in_das_OR}.
The term $C_{t}(S_{t},x_{t})$ represents a cost function that depends on the stage of the system $S_t$ and the chosen decision $x_t$, which is selected from the feasible set $\mathcal{X}_t$ of stage $t$.
The $\gamma$ represents a discount factor which describes how much the future outcomes are affecting present decisions \cite{Einfuehrung_in_das_OR, Powell_solving_Curses_of_Dimensionality}.
The discount factor can also be interpreted as a measure about how much the forecasts about the future outcomes are trusted. \\
After a decision $x_t$ is chosen and a realization from the underlying stochastic process can be observed, the environment transitions to the state $S_{t+1}$.
That state is determined by a transition function $S^M(S_t, x_t, \xi(\omega_t))$ \cite{Powell_solving_Curses_of_Dimensionality}.

Equation \ref{Bellman-Gleichung} shows that the multistage problem is decomposed into a here-and-now decision and a value function that accounts for future decisions that are affected by the current state decision.
The Bellman equation is then solved by backwards recursion, first solving the problem of stage $t=T$, where the value function for $t=T+1$ is usually set to $V_{T+1} \equiv 0$ for all states $S_T$.

Since the solution of the problem in stage $T$ is used as a value function in stage $T-1$, the problem for this stage can be solved.
The subsequent stages can then be solved following the same procedure.
This procedure can be visualized via the scenario tree in figure \ref{fig:scenario_tree}, by going backwards from the leafs to the root of the tree for all scenarios. \\
However, this approach quickly becomes computationally intractable, as the number of problems to solve increases exponentially with the number of stages. Furthermore, the probability distribution is assumed to be discrete and otherwise has to be discretized in the continuous case.
If the state space $\mathcal{S}_t$ for all states $S_t$ in stage $t$ is continuous or generally to large, the problem also becomes intractable since the value function has to be evaluated at every stage. 
If no closed-form or approximation of the value function is available, this task becomes nearly impossible \cite{Powell_solving_Curses_of_Dimensionality}. 
Generalizations of the Bellman equation to the continuous-time case are possible, the corresponding optimality condition is known as the Hamilton-Jacobi equation \cite{continous_time_stochastic_control}.
However, this will not be discussed further.

% The dynamic programming equations can be solved by SDP solution methods,
% backwards and evaluating the expected value functions Qt(·) for all possible states
% realizations of the uncertain data, which, in turn, requires to find an optimal decision over all possible actions xt.
% For this evaluation to be possible, it is assumed that the state space and the scenario space are finite – otherwise they have to be discretized.
% However, even in the discrete case, enumerating all possible combinations is com-
% putationally intractable for all but very low dimensions, as the number of evaluations
% suffers from combinatorial explosion. This phenomenon is known as the curse-of-
% dimensionality of SDP \cite{Powell_solving_Curses_of_Dimensionality}
\subsection{Challenges in Stochastic Optimization} \label{Challenges_stochastic_programming}
As described in the sections above, the main challenges in the two presented frameworks is that they do not scale very well for big problems, due to the curse of dimensionality.
One reason is that both approaches try to solve the respective sequential decision problem with a kind of backwards recursion. \\
This approach is rarely applicable for problem settings in the real world, since the computational effort increases drastically if the problem has more than two stages, a large state space and many possible realizations of the stochastic process \cite{Powell_solving_Curses_of_Dimensionality}. \\
The reason for this is the so-called \textit{curse of dimensionality}, a term coined by Bellman \cite{Bellman1957}.
It generally describes the challenge that when the dimension of a problem increases, exponentially many points are needed to continue to describe it completely \cite{Bellman1957}. \\
According to \cite{Powell_solving_Curses_of_Dimensionality}, \cite{Powell_Perspectives_of_ADP}, and \cite{TwenteBuchKapitel3}, the curse of dimensionality can be distinguished in three ways.
\begin{enumerate}
    \item The state space $\mathcal{S}$, i.e. the set of all states $S_t,\; t=1,\dots,T$ may become too large, such that it is no longer possible to evaluate the function $V_t(S_t)$ at each state $S_t$ in a reasonable time.
    \item The decision space $\mathcal{X}_t$ may be too large, so that no optimal or even good decision can be found for all states in reasonable time.
    \item The expected value of the value function may be too complex to be evaluated in reasonable time, since it may be a conditional expected value and there may be too many possibilities for realizations.
\end{enumerate}
Since problems encountered in practice very often have at least one of these three properties, dynamic optimization and hence stochastic optimization was long considered to be inapplicable in practice \cite{Powell_Clearing_the_Jungle_of_stochastic_Optimization}. \\
Sequential decision problems are nevertheless common, certain frameworks were developed to cope with the aforementioned curse of dimensionality.
One of which was the area of \textit{Approximate Dynamic Programming}, or ADP \cite{Powell_Perspectives_of_ADP} for short.
According to \cite{Powell_solving_Curses_of_Dimensionality}, this framework is used in many research communities, only under different names such as 
\textit{Neuro-Dynamic Programming} \cite{Neuro-dynamic-programming} or \textit{Reinforcement Learning} \cite{SuttonBartoRL}.
The goal here is to find an approximate solution to a dynamic optimization problem, since the curses of dimensionality make it very hard to find an exact solution in a reasonable amount of time. \\
The techniques developed in the stochastic programming community follow a similar approach, in that usually the expected value of the value function is approximated.
It can be argued that the methods used in stochastic programming can also be classified as methods of approximate dynamic programming \cite{POWELL2019795}.

In the following sections, a selection of these approximation methods is presented.
However, the focus will mainly be on the methods from stochastic programming.
